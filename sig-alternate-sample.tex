\documentclass{sig-alternate-05-2015}

\begin{document}

\title{Bioinformatics approaches to biomarker and drug discovery in aging and disease}
\numberofauthors{1}
\author{
\alignauthor
Amanda Wulf\\
       \affaddr{The University of Texas at San Antonio}\\
       \affaddr{1 UTSA Circle}\\
       \affaddr{San Antonio, Texas}\\
       \email{amandakwulf@gmail.com}
}
\maketitle
\begin{abstract}
Abstract goes here.
\end{abstract}

\section{Introduction}
% Big Picture
The development of high-throughput (HTP) technology has been a boon to
biology, allowing a lot more genes and biological data to be published.
Unfortunately, a lot of biological studies are published, but their datasets
tend to not stay
stable with each other. Biological data tends to be noisy and difficult to
analyze properly. These challenges can be mitigated by two methods: Meta-analyses in
order to bring more data to create better solutions, and improved analysis
methods.

% Past Work
Past studies in aging and disease tend to show inconsistent results due to poor
analysis methods and not enough diverse data.

% Idea behind solution
The projects in this dissertation serve to create better analysis methods and
do several meta-analyses to mitigate the issues inherent in HTP data.

% Results (especially compared to past works' results)
The results from these projects are all proven to be much more stable across
multiple datasets. In addition, many new discoveries have been made based on
data from diverse studies that was not analyzed together in the past.

% Conclusion
HTP data needs to continue to be analyzed in this manner in future studies and
meta-analyses.

\section{Related Works}
There are two main categories of computational biology studies that analyze
aging and disease: data integration and network analysis.

\subsection{Data Integration}
Data integration studies can either be meta-analyses that integrate the same
types of data across many studies like ..., or they can integrate several
different types of data into one analysis like ...

\subsection{Network Analysis}
This sort of analysis analyzes genes in tandem with other genes in so-called
molecular networks. ... is an example of this type of analysis.

\section{Project 1}
\textbf{"Inferring the functions of longevity genes with modular subnetwork biomarkers of Caenorhabditis elegans aging"}
\vspace{5mm}

This project develops a new method to identify gene expression biomarkers of
aging, using the topological modularity of the biological network. This method
yielded biomarkers that stayed the same across several studies, and was applied
to identify improved biomarkers and therapeutics for aging and disease-related
aging. It also was proven to accurately predict an organism's age based on its
gene expression data.

\section{Project 2}
\textbf{"In silico drug screen in mouse liver identifies candidate calorie restriction mimetics"}
\vspace{5mm}

This project develops a new meta-analysis pipeline that identifies drugs that
mimic the gene changes that occur as a result of calorie restriction in mice.
The new method discovered 14 drugs that consistently mimic calorie restriction
across gene signatures.

\section{Project 3}
\textbf{"NetwoRx: Connecting drugs to networks and phenotypes in S. cerevisiae"}
\vspace{5mm}

This project adapts gene-level high-throughput (HTP) chemogenomic data to study
drug response in yeast at the systems level and builds the NetwoRx portal in
order to make these data publically available.

\section{Project 4}
\textbf{"Computationally repurposing drugs for lung cancer with CMapBatch: candidate therapeutics from an integrative meta-analysis of cancer gene signatures and chemogenomic data"}
\vspace{5mm}

This project develops a meta-analysis pipeline to discover new drugs to reverse
pathological gene changes. The pipeline was applied to the pathological gene
changes associated with lung cancer and it produced a list of candidate drugs
to treat them. The candidate drugs produced by this method were proven to be
more stable than those produced by other methods.

\section{Conclusions}
Conclusion here.

\end{document}
